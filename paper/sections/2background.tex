\section{Background}
\label{sec:background}

In this section, we briefly explain the basic concept of Atomic Swaps and Option (in Finance), and their relationship.

\subsection{Atomic Swaps}
\TODO{More explanations on atomic swap.}

An Atomic Swap~\cite{nolan2013alt} is that two parties exchange their assets ``atomically'' without trusted third parties:
The swap either succeeds or fails for both parties.

In Blockchain, the Atomic Swap is realized by the Hashed Timelocked Contract (HTLC)~\cite{poon2016bitcoin}. The HTLC was originally introduced to secure routing across multiple payment channels~\cite{paychannel2018btcwiki} - which aim at accerelating up trasactions by allowing users to make multiple transactions without commiting all of them.
In HTLC-style transactions, the payee redeems the payment prior to a deadline by providing the preimage of a specific hash value.
Otherwise, the payment will expire and the money goes back to the payer.
HTLC is based on the hashlock - to lock the payment by a hash value, and the timelock - to limit the deadline for the payee to redeem.
The Atomic Swap protocol can be implemented on all blockchains supporting the timelock and the hashlock.

\subsection{Option}
\label{subsec:background_option}

% option
In finance, an option is a contract which gives the option buyer the right to buy or sell an asset, at a specified price prior to or on a specified date~\cite{higham2004introduction}.
Here the option buyer can choose whether to fulfill the contract.
The specified price is called the \textit{strike price};
the specified date is called the \textit{strike time};
the party proposing the option is called the \textit{option seller};
the other party is called the \textit{option buyer};
the asset is called the \textit{underlying asset};
and fulfilling the contract is called \textit{exercising}.

% american option and european option
The option has two types: The American-style Option and the European-style Option.
They differ from the \textit{strike time}:
The European-style Option buyer can only exercise the contract on the strike time,
and the American-style Option buyer can exercise the contract prior to or on the strike time.

% call option and put option
If the option buyer is buying the underlying asset, the option is called the Call Option, otherwise it is called the Put Option.

% what is premium
Besides the underlying asset, the option contract itself is considered to have value.
The value of the contract is called the \textit{premium}.
The option buyer should pay for the \textit{premium} to the option seller besides exchanging the underlying asset.

% option pricing
The \textit{premium} is priced prior to the contract agreement.
As the \textit{premium} is the only variable within the option contract,
pricing the \textit{premium} is also known as the \textit{option pricing} problem.
\textit{Option Pricing} is rather a complex task - It is still a hot research topic in Finance and Applied Mathematics.

The Black-Scholes (BS) Model is the first widely used model for option pricing~\cite{black1973pricing}.
It can estimate the theoretical value of European-style Options using historical price data of the underlying asset.
The Cox-Ross-Rubinstein (CRR) model~\cite{cox1979option}, also known as the Binomial Option Pricing model, extends the BS model for pricing American-style Options.

\subsection{Optionality of Atomic Swaps}

\TODO{More deep observations on optionality}
As the Atomic Swap is not instant, it introduces the optionality - The contract is settled after the contract agreement.
However, the Atomic Swap is designed for currency exchange rather than for implementing options.
We will prove that the Atomic Swap is equivalent to the American Call Option
where the underlying asset strike price is the current strike price in Section~\ref{sec:formalization}.