\section{Deployment}
\label{sec:deployment}

\HY{DEPLOYMENT happens after DEVELOPMENT. maybe you mean implementation?}

\TODO{intro, and make this section Implementation}
\TODO{figure ref ????}

\subsection{Requirements}

To implement our protocols, the blockchain should support the following functionalities:

\begin{enumerate}
    \item Stateful transactions
    \item Timelock
    \item Hashlock
\end{enumerate}

\paragraph{Stateful transactions}
Transactions should be stateful: Executing a transaction can depend on prior transactions.
In our protocols, whether $pr$ goes to Alice or Bob depends on the status of $Coin_2$.
Therefore, the transaction of $pr$ relies on the status of $Coin_2$ payment transaction.

\paragraph{Hashlock}
The transactions should support the hashlock: A payment is proceeded only when the payee provides the preimage of a hash.
In our protocols, exchanging $Coin_1$ and $Coin_2$ atomically is based on the hashlock -
Alice redeems $Coin_2$ first by releasing the preimage, then Bob can redeem $Coin_1$ by using the released preimage.

\paragraph{Timelock}
The transactions should support the timelock: A payment will expire after a specified time if the payee cannot redeem the payment.
Note that the payment can be conditional, e.g. by hashlock.
In our protocols, the transactions of $Coin_1$, $Coin_2$ and $pr$ are all timelocked, in order to avoid locking money in transactions forever.

\subsection{Smart contracts}

Smart contracts support all aforementioned functionalities, so can easily implement our protocols.
We use the Solidity - the programming language for Ethereum smart contracts~\cite{wood2014ethereum} - as an example.

Our implementations are based on the original Atomic Swap Solidity implementation~\footnote{\url{https://github.com/AltCoinExchange/ethatomicswap}.},
but extend the premium mechanism.

Extending the premium mechanism includes:

\begin{enumerate}
    \item The enumeration \textit{AssetState} for maintaining the premium payment state
    \item The modifiers \textit{isPremiumRedeemable()} and \textit{isPremiumRefundable()} for checking whether the premium can be redeemed or refunded
    \item The methods \textit{redeemPremium()} and \textit{refundPremium()} for redeeming and refunding the premium
\end{enumerate}

\paragraph{The premium payment state \textit{AssetState}}

Similar with the \textit{PremiumState}, \textit{AssetState} has 4 states: empty, filled, redeemded and refunded.
The code is shown in Figure~\ref{code:state}.
\textit{Empty} means Alice has not performed \textbf{initiate}, and has not deposited the premium yet.
\textit{Filled} means Alice has deposited the premium, indicating that Alice has performed \textbf{initiate}, but neither Alice nor Bob \textbf{refunds} or \textbf{redeems} the premium.
\textit{Redeemded} and \textit{refunded} means Bob redeems the premium and Alice refunds the premium, respectively.

\begin{figure}
\begin{lstlisting}[language=Solidity, basicstyle=\tiny]
enum AssetState { Empty, Filled, Redeemed, Refunded }
enum PremiumState { Empty, Filled, Redeemed, Refunded }
\end{lstlisting}
\label{code:state}
\caption{Maintaining the state of the asset and the premium.}
\end{figure}

\paragraph{\textit{isPremiumRedeemable()} and \textit{isPremiumRefundable()}}
Checking whether the premium is redeemable or refundable is the most critical part of our protocols.
Because the premium payment relies on the $Coin_2$ payment, checking the premium refundability and redeemability involves checking the $Coin_2$ status - \textit{AssetState} in our implementation.

\textit{isPremiumRedeemable()} and \textit{isPremiumRefundable()} for the currency exchange-style Atomic Swap are shown in Figure~\ref{code:premium_condition_currency}, and for the American Call Option-style Atomic Swap are shown in Figure~\ref{code:premium_condition_options}.
The currency exchange-style Atomic Swap and the American Call Option-style Atomic Swap differ when $AssetState = Redeemed$:
In the currency exchange-style Atomic Swap the premium belongs to Alice while in the American Call Option-style Atomic Swap the premium belongs to Bob.

\begin{figure}
\begin{lstlisting}[language=Solidity, basicstyle=\tiny]
// Premium is refundable when
// 1. Alice initiates but Bob does not participate
//   after premium's timelock expires
// 2. asset2 is redeemed by Alice
modifier isPremiumRefundable(bytes32 secretHash) {
    // the premium should be deposited
    require(swaps[secretHash].premiumState == PremiumState.Filled);
    // the initiator invokes this method to refund the premium
    require(swaps[secretHash].initiator == msg.sender);
    // the contract should be on the blockchain2
    require(swaps[secretHash].kind == Kind.Participant);
    // if the asset2 timelock is still valid
    if block.timestamp <= swaps[secretHash].assetRefundTimestamp {
        // the asset2 should be redeemded by Alice
        require(swaps[secretHash].assetState == AssetState.Redeemed);
    } else { // if the asset2 timelock is still valid
        // the asset2 should not be refunded
        require(swaps[secretHash].assetState != AssetState.Refunded);
        // the premium timelock should be expired
        require(block.timestamp > swaps[secretHash].premiumRefundTimestamp);
    }
    _;
}
// Premium is redeemable for Bob when asset2 is refunded
// which means Alice holds the secret maliciously
modifier isPremiumRedeemable(bytes32 secretHash) {
    // the premium should be deposited
    require(swaps[secretHash].premiumState == PremiumState.Filled);
    // the participant invokes this method to redeem the premium
    require(swaps[secretHash].participant == msg.sender);
    // the contract should be on the blockchain2
    require(swaps[secretHash].kind == Kind.Participant);
    // the asset2 should be refunded
    // this also indicates the asset2 timelock is expired
    require(swaps[secretHash].assetState == AssetState.Refunded);
    // the premium timelock should not be expired
    require(block.timestamp <= swaps[secretHash].premiumRefundTimestamp);
    _;
}
\end{lstlisting}
\label{code:premium_condition_currency}
\caption{The condition to redeem and refund the premium for currency-exchange-style Atomic Swaps.}
\end{figure}

\begin{figure}
\begin{lstlisting}[language=Solidity, basicstyle=\tiny]
// Premium is refundable for Alice only when Alice initiates
// but Bob does not participate after premium's timelock expires
modifier isPremiumRefundable(bytes32 secretHash) {
    // the premium should be deposited
    require(swaps[secretHash].premiumState == PremiumState.Filled);
    // the initiator invokes this method to refund the premium
    require(swaps[secretHash].initiator == msg.sender);
    // the contract should be on the blockchain2
    require(swaps[secretHash].kind == Kind.Participant);
    // premium timelock should be expired
    require(block.timestamp > swaps[secretHash].premiumRefundTimestamp);
    // asset2 should be empty
    // which means Bob does not participate
    require(swaps[secretHash].assetState == AssetState.Empty);
}
// Premium is redeemable for Bob when asset2 is redeemed or refunded
// which means Bob participates
modifier isPremiumRedeemable(bytes32 secretHash) {
    // the premium should be deposited
    require(swaps[secretHash].premiumState == PremiumState.Filled);
    // the participant invokes this method to redeem the premium
    require(swaps[secretHash].participant == msg.sender);
    // the contract should be on the blockchain2
    require(swaps[secretHash].kind == Kind.Participant);
    // the asset2 should be refunded or redeemed
    require(swaps[secretHash].assetState == AssetState.Refunded || swaps[secretHash].assetState == AssetState.Redeemed);
    // the premium timelock should not be expired
    require(block.timestamp <= swaps[secretHash].premiumRefundTimestamp);
    _;
}
\end{lstlisting}
\label{code:premium_condition_options}
\caption{The condition to redeem and refund the premium for American Call Option-style Atomic Swaps.}
\end{figure}



\paragraph{\textit{redeemPremium()} and \textit{refundPremium()}}

\textit{redeemPremium()} and \textit{refundPremium()} are similar with \textit{redeemAsset()} and \textit{refundAsset()}, and their executions are secured by \textit{isPremiumRedeemable()} and \textit{isPremiumRefundable()}.
The code is shown in Figure~\ref{code:premium_redeem_refund_function}.

\begin{figure}
\begin{lstlisting}[language=Solidity, basicstyle=\tiny]
function redeemPremium(bytes32 secretHash)
    public
    isPremiumRedeemable(secretHash)
{
    // transfer the premium to Bob
    swaps[secretHash].participant.transfer(swaps[secretHash].premiumValue);
    // update the premium state to redeemded
    swaps[secretHash].premiumState = PremiumState.Redeemed;
    // notify the function invoker
    emit PremiumRefunded(
        block.timestamp,
        swaps[secretHash].secretHash,
        msg.sender,
        swaps[secretHash].premiumValue
    );
}
    function refundPremium(bytes32 secretHash)
    public
    isPremiumRefundable(secretHash)
{
    // transfer the premium to Alice
    swaps[secretHash].initiator.transfer(swaps[secretHash].premiumValue);
    // update the premium state to refunded
    swaps[secretHash].premiumState = PremiumState.Refunded;
    // notify the function invoker
    emit PremiumRefunded(
        block.timestamp,
        swaps[secretHash].secretHash,
        msg.sender,
        swaps[secretHash].premiumValue
    );
}
\end{lstlisting}
\label{code:premium_redeem_refund_function}
\caption{The functions for redeeming and refunding the premium.}
\end{figure}


\subsection{Bitcoin script}

% bitcoin cannot implement
% because stateless
Unfortunately, Bitcoin script does not support our protocols, because the Bitcoin script does not support the stateful transaction functionalities - it is designed to be stateless. \HY{Maybe explain why itz stateless? e.g., therez no such a thing as "world state"; and nothing about the state is stored?}

% new opcode
\paragraph{New Opcode OP\_LOOKUP\_OUTPUT}
In order to make Bitcoin script support our protocols, we propose a new opcode called OP\_LOOKUP\_OUTPUT.
OP\_LOOKUP\_OUTPUT takes the id of an output, and produces the address of the output's owner.
With OP\_LOOKUP\_OUTPUT, the Bitcoin script can decide whether Alice or Bob should take the premium by  \textbf{<asset2\_output> OP\_LOOKUP\_OUTPUT <Alice\_pubkeyhash> OP\_EQUALVERIFY}. \HY{The printed paragraph doesn't fit to the page margin, at least in mine.}

% implement opcode is easy
Implementing OP\_LOOKUP\_OUTPUT is easy in Bitcoin - It only queries the ownership of an output from the indexed blockchain database, and does not require any computation.

\paragraph{Decoupling the contract creation and the contract invocation}
For smart contracts, the contract is created and invoked in separate transactions.
However, for Bitcoin script, the contract is created and invoked in the same transaction, which makes our protocols difficult to construct.
First, the timelock starts right after the contract creation rather than the contract invocation.
Second, the premium does not know which outputs belong to $Coin_2$.

We solve this problem by constructing the contract offline:
Alice and Bob create a 2-2 multisignature address, and merge the premium payment and the $Coin_2$ payment into a single transaction.
First, Alice and Bob create a 2-2 multisignature address.
In order to use the address to make a payment, both Alice and Bob should sign the transaction.
Second, Bob creates the $Coin_2$ transaction (offline) which includes the unspent outputs, and sends it to Alice.
Third, Alice adds the premium transaction to Bob's sent transaction.
The premium transaction uses OP\_LOOKUP\_OUTPUT to check the ownership of outputs of $Coin_2$.
Finally, both Alice and Bob sign and publish the transaction.

\begin{figure}
\begin{lstlisting}[language=Solidity, basicstyle=\tiny]
ScriptSig:
    Redeem: <Bob_sig> <Bob_pubkey> 1
    Refund: <Alice_sig> <Alice_pubkey> 0
ScriptPubKey:
    OP_IF // Normal redeem path
        // the owner of <asset2_output> should be Alice
        // which means Alice has redeemed asset2
        <asset2_output> OP_LOOKUP_OUTPUT <Alice_pubkeyhash> OP_EQUALVERIFY 
        OP_DUP OP_HASH160 <Bob_pubkeyhash>
    OP_ELSE // Refund path
        // the premium timelock should be expired
        <locktime> OP_CHECKLOCKTIMEVERIFY OP_DROP
        OP_DUP OP_HASH160 <Alice pubkey hash>
    OP_ENDIF
    OP_EQUALVERIFY
    OP_CHECKSIG
\end{lstlisting}
\label{code:bitcoin_contract_currency_exchange}
\caption{The currency exchange-style Atomic Swap contract in Bitcoin script}
\end{figure}

\begin{figure}
\begin{lstlisting}[language=Solidity, basicstyle=\tiny]
ScriptSig:
    Redeem: <Bob_sig> <Bob_pubkey> 1
    Refund: <Alice_sig> <Alice_pubkey> 0
ScriptPubKey:
    OP_IF // Normal redeem path
        // the owner of the asset2 should not be the contract
        // it should be either (redeemde by) Alice or (refunded by) Bob
        // which means Alice has redeemed asset2
        <asset2_output> OP_LOOKUP_OUTPUT <Alice_pubkeyhash> OP_NUMEQUAL
        <asset2_output> OP_LOOKUP_OUTPUT <Bob_pubkeyhash> OP_NUMEQUAL
        OP_ADD 1 OP_NUMEQUALVERIFY
        OP_DUP OP_HASH160 <Bob_pubkeyhash>
    OP_ELSE // Refund path
        // the premium timelock should be expired
        <locktime> OP_CHECKLOCKTIMEVERIFY OP_DROP
        OP_DUP OP_HASH160 <Alice pubkey hash>
    OP_ENDIF
    OP_EQUALVERIFY
    OP_CHECKSIG
\end{lstlisting}
\label{code:bitcoin_contract_option}
\caption{The American Call Option-style Atomic Swap contract in Bitcoin script}
\end{figure}

Figure~\ref{code:bitcoin_contract_currency_exchange} and Figure~\ref{code:bitcoin_contract_option} show the Bitcoin scripts for the currency-style and the American Call Option-style Atomic Swaps, respectively.

% OP_LOOKUP_OUTPUT is from https://bitcoin.stackexchange.com/questions/36229/bitcoin-script-for-a-competitive-crowdfunding-like-contract