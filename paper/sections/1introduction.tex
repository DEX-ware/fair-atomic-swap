\section{Introduction}
\label{sec:intro}

The Atomic Swap protocols allow two parties to exchange their assets ``atomically'' without trusted third parties.
``Atomic'' means the swap either succeeds or fails for both parties at any given time.
In blockchains, Atomic Swap is usually implemented by using Hashed Time-locked Contracts (HTLCs)~\cite{poon2016bitcoin}.
The HTLC is a type of transaction that, the payee should provide the preimage of a hash value before a specified deadline, otherwise the payment fails - the money goes back to the payer and the payee will not get any money.

% how important is Atomic Swap
Atomic Swap has already been widely adopted in the cryptocurrency industry.
In particular, it realised the Decentralised Exchanges (DEXes), and boosted the Decentralised Finance (De-Fi).
In a DEX, traders publish and participate in deals, and the dealing process is regulated by a smart contract embedding the Atomic Swap protocol.
Different from centralised exchanges, DEXes are non-custodial - traders does not need to deposit their money in DEXes.
The non-custody property avoids DEXes from money thefts, which commonly happened on centralised exchanges.
To date, the DEX market volume has reached approximately 50,000 ETH~\cite{dexwatch}.
More specifically,
there are more than 250 DEXes~\cite{distribuyed/index},
more than 30 DEX protocols~\cite{evbots/dex-protocols},
and more than 4,000 active traders in all DEXes~\cite{dexwatch}.



% research problem
However, being atomic does not indicate the Atomic Swap is fair.
In an Atomic Swap, the swap initiator can decide whether to proceed or abort the swap, and the default maximum time for him to decide is 24 hours~\cite{nolan2013alt}.
This enables the the swap initiator to speculate without any penalty.
More specifically, the swap initiator can keep waiting before the timelock expires.
If the price of the swap participant's asset rises, the swap initiator will proceed the swap so that he will profit.
If the price of the swap participant's asset drops, the swap initiator can abort the swap, so that he won't lose money.

% optionality
ZmnSCPxj~\cite{optionality-origin} observed that, this problem is equivalent to the Optionality in Finance, which has already been studied for decades~\cite{higham2004introduction}.
In Finance, an investment is said to have Optionality if
1) settling this investment happens in the future rather than instantly;
2) settling this investment is optional rather than mandatory.
For an investment with Optionality, the option itself has value besides the underlying asset, which is called the \textit{premium}.
The option buyer should pay for the premium besides the underlying asset, even if he aborts the contract.
In this way, he can no longer speculate without penalties.

% initiator in atomic swap has optionality
In Atomic Swap, the swap initiator has the Optionality, as he can choose whether to proceed or abort the swap.
Unfortunately, the swap initiator is not required to pay for the premium - the Atomic Swap does not take the Optionality into account.
% atomic swap should not have optionality
Furthermore, Atomic Swap should not have Optionality.
Atomic Swap is designed for currency exchange, and the currency exchange has no Optionality.
Instead, once both parties agree on a currency exchange, it should be settled without any chance to regret.

% what we do briefly
In this paper, we investigate the unfairness of Atomic Swap.
We start from describing the Atomic Swap and the American Call Option in Finance,
then we show how an Atomic Swap is equivalent to a premium-free American Call Option.
After that, we then evaluate how unfair the Atomic Swap is from two different perspectives:
quantifying the unfairness and estimating the premium.
Furthermore, we propose an improvement on the Atomic Swap, which implements the premium mechanism, to make it fair.
Our improvement supports blockchains with smart contracts (e.g. Ethereum) directly, and can support blockchains with scripts only (e.g. Bitcoin) by adding a single opcode.
We also implement our protocol in Solidity (a smart contract programming language for Ethereum), and give detailed instructions on implementing our protocols on Bitcoin.

\subsection{Our contributions}

Our contributions are as follows:

\paragraph{We quantify the unfairness of Atomic Swap, and compare it with that of conventional financial assets.}
Following ZmnSCPxj's observation~\cite{optionality-origin} that the Atomic Swap is equivalent to the premium-free American Call Option,
we quantify how unfair the Atomic Swap with mainstream cryptocurrency pairs is, and compare this unfairness with those of conventional financial assets (stocks and fiat currencies).
We first classify the unfairness to two parts, namely the profit when the price rises and the mitigated loss when the price drops, then quantify them based on historical exchange rate volatility.
Our results show that, in the default timelock setting, the profit and the mitigated loss of our selected cryptocurrency pairs are approximately 1\%, while for stocks and fiat currencies the values are approximately 0.3\% and 0.15\%, respectively.

\paragraph{We use the Cox-Ross-Rubinstein option pricing model to estimate the premium.}
We use the Cox-Ross-Rubinstein option pricing model to estimate how much the premium should be for Atomic Swaps.
In Finance, the Cox-Ross-Rubinstein model~\cite{cox1979option} is the conventional option pricing model for American-style options.
Our results show that, in the default timelock setting, the premium should be approximately 2\% for Atomic Swaps with cryptocurrency pairs, while the premium is approximately $0.3\%$ for American Call Options with stocks and fiat currencies.
Also, the premium values rise for all assets with the strike time increasing, then start to converge when the strike time reaches 300 days.

\paragraph{We propose an improvement on the Atomic Swap to make it fair.}
With the observation that the unfairness is from the premium,
we propose an improvement on the Atomic Swap, which implements the premium mechanism, to make it fair.
It supports both the currency exchange-style Atomic Swap and the American Call Option-style Atomic Swap.
In the currency exchange-style Atomic Swap, the premium will go back to the swap initiator if the swap is successful.
In the American Call Option-style Atomic Swap, the premium will definitely go to the swap participant if the participant participates in the swap.

\paragraph{We describe how to implement our protocol on existing blockchains.}
We give instructions to implement our protocols on existing blockchains,
including blockchains supporting smart contracts and blockchains supporting scripts only.
For blockchains supporting smart contracts (e.g. Ethereum), our protocol can be directly implemented.
For blockchains supporting scripts only (e.g. Bitcoin), our protocol can be implemented by adding one more opcode.
We call the opcode ``OP\_LOOKUP\_OUTPUT'', which looks up the owner of a specific UTXO output.
We give the reference implementation in Solidity as an example of smart contracts.
We also give that in Bitcoin script (which assumes ``OP\_LOOKUP\_OUTPUT'' exists) as an example of scripts.










\subsection{Paper structure}

The paper is structured as follows.
Section~\ref{sec:background} describes the background of Atomic Swap and options in Finance.
Section~\ref{sec:formalization} describes the Atomic Swap and the American Call Option, and shows how the Atomic Swap protocol is equivalent to the premium-free American Call Option.
Section~\ref{sec:evaluation} evaluates the Atomic Swap unfairness by analysing the volatility and pricing the premium of mainstream cryptocurrency pairs.
Section~\ref{sec:fair_atomic_swap} describes our proposed fair Atomic Swap protocols.
Section~\ref{sec:implementation} describes how to implement our proposed protocols on existing blockchains.
Section~\ref{sec:discussion} discusses security issues of Atomic Swaps, other countermeasures for solving the Atomic Swap unfairness, and limitations of our protocols.
Section~\ref{sec:conclusion} concludes our paper and outlines the future work.
