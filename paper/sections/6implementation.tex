\section{Implementation}
\label{sec:implementation}

In this section, we describe how to implement our proposed protocol in Section~\ref{sec:fair_atomic_swap} on different blockchains, including blockchains only supporting scripts (such as Bitcoin) and blockchains supporting smart contracts (such as Ethereum).
In particular, we describe our design rationale, and provide reference implementations in Bitcoin scripts and Solidity smart contracts.

\subsection{Requirements}

To implement our protocol, the blockchain should support 1) stateful transactions, 2) the timelock and 3) the hashlock.

\paragraph{Stateful transactions}
Transactions should be stateful: executing a transaction can depend on prior transactions.
In our protocol, whether $pr$ goes to Alice or Bob depends on the status of $Coin_2$.
Therefore, the transaction of $pr$ relies on the status of $Coin_2$ payment transaction.

\paragraph{Hashlock}
The transactions should support the hashlock: a payment is proceeded only when the payee provides the preimage of a hash.
In our protocol, exchanging $Coin_1$ and $Coin_2$ atomically is based on the hashlock -
Alice redeems $Coin_2$ first by releasing the preimage, then Bob can redeem $Coin_1$ by using the released preimage.

\paragraph{Timelock}
The transactions should support the timelock: a payment will expire after a specified time if the payee cannot redeem the payment.
In our protocol, the transactions of $Coin_1$, $Coin_2$ and $pr$ are all timelocked, in order to avoid locking money in transactions forever.

\subsection{Smart contracts}

Smart contracts support all aforementioned functionalities, so can easily implement our protocol.
We use Solidity - one of programming languages for Ethereum smart contracts~\cite{wood2014ethereum} - as an example.
Our implementations are based on the original Atomic Swap Solidity implementation~\cite{AltCoinExchange/ethatomicswap},
but extend the premium mechanism.
Extending the premium mechanism includes:

\begin{enumerate}
    \item The enumeration \textit{PremiumState} for maintaining the premium payment state
    \item The modifiers \textit{isPremiumRedeemable()} and \textit{isPremiumRefundable()} for checking whether the premium can be redeemed or refunded
    \item The methods \textit{redeemPremium()} and \textit{refundPremium()} for redeeming and refunding the premium
\end{enumerate}

\begin{figure}[htp]
\begin{lstlisting}[
    language=Solidity, 
    basicstyle=\tiny,
    label={lst:state},
    caption={Maintaining the state of the asset and the premium.}
]
enum AssetState { Empty, Filled, Redeemed, Refunded }
enum PremiumState { Empty, Filled, Redeemed, Refunded }
\end{lstlisting}
\end{figure}

\begin{figure}[htp]
\begin{lstlisting}[
    language=Solidity, 
    basicstyle=\tiny,
    label={lst:premium_condition_currency},
    caption={The condition to redeem and refund the premium for currency-exchange-style Atomic Swaps.}
]
// Premium is refundable when
// 1. Alice initiates but Bob does not participate
//   after premium's timelock expires
// 2. asset2 is redeemed by Alice
modifier isPremiumRefundable(bytes32 secretHash) {
    // the premium should be deposited
    require(swaps[secretHash].premiumState == PremiumState.Filled);
    // Alice invokes this method to refund the premium
    require(swaps[secretHash].initiator == msg.sender);
    // the contract should be on the blockchain2
    require(swaps[secretHash].kind == Kind.Participant);
    // if the asset2 timelock is still valid
    if block.timestamp <= swaps[secretHash].assetRefundTimestamp {
        // the asset2 should be redeemded by Alice
        require(swaps[secretHash].assetState == AssetState.Redeemed);
    } else { // if the asset2 timelock is expired
        // the asset2 should not be refunded
        require(swaps[secretHash].assetState != AssetState.Refunded);
        // the premium timelock should be expired
        require(block.timestamp > swaps[secretHash].premiumRefundTimestamp);
    }
    _;
}
// Premium is redeemable for Bob when asset2 is refunded
// which means Alice holds the secret maliciously
modifier isPremiumRedeemable(bytes32 secretHash) {
    // the premium should be deposited
    require(swaps[secretHash].premiumState == PremiumState.Filled);
    // Bob invokes this method to redeem the premium
    require(swaps[secretHash].participant == msg.sender);
    // the contract should be on the blockchain2
    require(swaps[secretHash].kind == Kind.Participant);
    // the asset2 should be refunded
    // this also indicates the asset2 timelock is expired
    require(swaps[secretHash].assetState == AssetState.Refunded);
    // the premium timelock should not be expired
    require(block.timestamp <= swaps[secretHash].premiumRefundTimestamp);
    _;
}
\end{lstlisting}
\end{figure}

\begin{figure}[htp]
\begin{lstlisting}[
    language=Solidity, 
    basicstyle=\tiny,
    label={lst:premium_condition_options},
    caption={The condition to redeem and refund the premium for American Call Option-style Atomic Swaps.}
]
// Premium is refundable for Alice only when Alice initiates
// but Bob does not participate after premium's timelock expires
modifier isPremiumRefundable(bytes32 secretHash) {
    // the premium should be deposited
    require(swaps[secretHash].premiumState == PremiumState.Filled);
    // Alice invokes this method to refund the premium
    require(swaps[secretHash].initiator == msg.sender);
    // the contract should be on the blockchain2
    require(swaps[secretHash].kind == Kind.Participant);
    // premium timelock should be expired
    require(block.timestamp > swaps[secretHash].premiumRefundTimestamp);
    // asset2 should be empty
    // which means Bob does not participate
    require(swaps[secretHash].assetState == AssetState.Empty);
}
// Premium is redeemable for Bob when asset2 is redeemed or refunded
// which means Bob participates
modifier isPremiumRedeemable(bytes32 secretHash) {
    // the premium should be deposited
    require(swaps[secretHash].premiumState == PremiumState.Filled);
    // Bob invokes this method to redeem the premium
    require(swaps[secretHash].participant == msg.sender);
    // the contract should be on the blockchain2
    require(swaps[secretHash].kind == Kind.Participant);
    // the asset2 should be refunded or redeemed
    require(swaps[secretHash].assetState == AssetState.Refunded || swaps[secretHash].assetState == AssetState.Redeemed);
    // the premium timelock should not be expired
    require(block.timestamp <= swaps[secretHash].premiumRefundTimestamp);
    _;
}
\end{lstlisting}
\end{figure}

\paragraph{The premium payment state \textit{PremiumState}}

In the original smart contract, an enumeration \textit{State} maintains the asset state:
\textbf{empty} means the asset has not been deposited;
\textbf{filled} means the asset has been deposited;
\textbf{redeemed} means the asset has been redeemed;
\textbf{refunded} means the asset has been refunded.

In our contract, we decouple \textit{State} to the asset state \textit{AssetState} and the premium state \textit{PremiumState}.
Both \textit{AssetState} and \textit{PremiumState} are the same as the original \textit{State}.
The code is shown in Listing~\ref{lst:state}.
\textit{Empty} means Alice has not triggered \textbf{Initiate}, and has not deposited the premium yet.
\textit{Filled} means Alice has deposited the premium, indicating that Alice has triggered \textbf{Initiate}, but neither Alice nor Bob refunds or redeems the premium.
\textit{Redeemded} and \textit{refunded} means Bob redeems the premium and Alice refunds the premium, respectively.

\paragraph{\textit{isPremiumRedeemable()} and \textit{isPremiumRefundable()}}
Checking whether the premium is redeemable or refundable is the most critical part of our protocol.
Because the premium payment relies on the $Coin_2$ payment, checking the premium refundability and redeemability involves checking the $Coin_2$ status - \textit{AssetState} in our implementation.

\textit{isPremiumRedeemable()} and \textit{isPremiumRefundable()} for the currency exchange-style Atomic Swap are shown in Figure~\ref{lst:premium_condition_currency}, and for the American Call Option-style Atomic Swap are shown in Figure~\ref{lst:premium_condition_options}.
The currency exchange-style Atomic Swap and the American Call Option-style Atomic Swap differ when $AssetState = Redeemed$:
in the currency exchange-style Atomic Swap the premium belongs to Alice while in the American Call Option-style Atomic Swap the premium belongs to Bob.

\paragraph{\textit{redeemPremium()} and \textit{refundPremium()}}

\textit{redeemPremium()} and \textit{refundPremium()} are similar to \textit{redeemAsset()} and \textit{refundAsset()}, and their executions are secured by \textit{isPremiumRedeemable()} and \textit{isPremiumRefundable()}.
The code is shown in Listing~\ref{lst:premium_redeem_refund_function}.

\begin{figure}[htp]
\begin{lstlisting}[
    language=Solidity, 
    basicstyle=\tiny,
    label={lst:premium_redeem_refund_function},
    caption={The functions for redeeming and refunding the premium.}
]
function redeemPremium(bytes32 secretHash)
    public
    isPremiumRedeemable(secretHash)
{
    // transfer the premium to Bob
    swaps[secretHash].participant.transfer(swaps[secretHash].premiumValue);
    // update the premium state to redeemded
    swaps[secretHash].premiumState = PremiumState.Redeemed;
    // notify the function invoker
    emit PremiumRefunded(
        block.timestamp,
        swaps[secretHash].secretHash,
        msg.sender,
        swaps[secretHash].premiumValue
    );
}
function refundPremium(bytes32 secretHash)
    public
    isPremiumRefundable(secretHash)
{
    // transfer the premium to Alice
    swaps[secretHash].initiator.transfer(swaps[secretHash].premiumValue);
    // update the premium state to refunded
    swaps[secretHash].premiumState = PremiumState.Refunded;
    // notify the function invoker
    emit PremiumRefunded(
        block.timestamp,
        swaps[secretHash].secretHash,
        msg.sender,
        swaps[secretHash].premiumValue
    );
}
\end{lstlisting}
\end{figure}

\subsection{Bitcoin script}

% bitcoin cannot implement
% because stateless
Unfortunately, Bitcoin cannot support our protocol directly, because Bitcoin does not support the stateful transaction functionalities.
First, the Bitcoin script is designed to be stateless~\cite{okupski2014bitcoin}.
Second, there is no such things like the Ethereum's ``world state''~\cite{wood2014ethereum} in Bitcoin:
the only state in Bitcoin is the Unspent Transaction Outputs (UTXOs)~\cite{nakamoto2008bitcoin}.


% new opcode
\paragraph{New Opcode OP\_LOOKUP\_OUTPUT}
In order to make Bitcoin script support our protocol, we use an opcode called OP\_LOOKUP\_OUTPUT.
OP\_LOOKUP\_OUTPUT was proposed, but has not implemented in Bitcoin yet~\cite{op-lookup-output-origin}.
It takes the id of an output, and produces the address of the output's owner.
With OP\_LOOKUP\_OUTPUT, the Bitcoin script can decide whether Alice or Bob should take the premium by
``<asset2\_output> OP\_LOOKUP\_OUTPUT <Alice\_pubkeyhash> OP\_EQUALVERIFY''.

% implement opcode is easy
Implementing OP\_LOOKUP\_OUTPUT is easy in Bitcoin - it only queries the ownership of an output from the indexed blockchain database.
This neither introduces computation overhead, nor breaks the ``stateless'' design of the Bitcoin script.


\paragraph{Decoupling the contract creation and the contract invocation}
For smart contracts, the contract is created and invoked in separate transactions:
creating the contract is by publishing a transaction which creates the smart contract,
and invoking the contract is by publishing a transaction which invokes a method in the smart contract.
However, Bitcoin has no smart contracts, and the ``contract'' is created and invoked in a single transaction.
In this way, the timelock starts right after the contract creation rather than the contract invocation.
This is problematic: the premium contract should not be triggered until Bob participates in the swap.

Thanks to the multi-signature transaction functionality in Bitcoin,
Alice and Bob can first create the contract off-chain, then invoke the contract on-chain.

Multi-signature transactions refer to transactions signed by multiple accounts~\cite{okupski2014bitcoin}.
A M-of-N ($M \leq N$) multi-signature transaction means the transaction requires $M$ out of $N$ accounts to sign it.
If less than $M$ accounts sign the transaction, the transaction cannot be verified as valid by the blockchain.
In Bitcoin, constructing a multi-signature transaction requires accounts to create a multi-signature address first~\cite{okupski2014bitcoin}.

With multi-signature transactions, we can decouple the contract creation and invocation as follows:
first, Alice and Bob create a 2-of-2 multi-signature address;
second, Alice and Bob mutually construct and sign a transaction which includes the premium payment and the $Coin_2$ payment;
finally, they publish the transaction in the name of the 2-2 multi-signature address.

Note that constructing and signing the transaction is done off-chain:
first, Bob creates the $Coin_2$ transaction and sends it to Alice;
second, Alice creates the premium transaction which uses OP\_LOOKUP\_OUTPUT to check the ownership of $Coin_2$ transaction outputs;
third, Alice merges the $Coin_2$ transaction and the premium transaction to a single transaction, signs the transaction, and sends it to Bob;
finally, Bob signs the transaction and sends it to Alice.
At this stage, both Alice and Bob have obtained the mutually signed transaction, which consists of both the premium transaction and the $Coin_2$ transaction.

\paragraph{The premium transaction}
Listing~\ref{lst:bitcoin_contract_currency_exchange} and Listing~\ref{lst:bitcoin_contract_option} show the premium transaction in the Bitcoin script
, for both the currency-style and the American Call Option-style Atomic Swaps, respectively.

\begin{figure}[htp]
\begin{lstlisting}[
    language=Solidity, 
    basicstyle=\tiny,
    label={lst:bitcoin_contract_currency_exchange},
    caption={The currency exchange-style Atomic Swap contract in Bitcoin script.}
]
ScriptSig:
    Redeem: <Bob_sig> <Bob_pubkey> 1
    Refund: <Alice_sig> <Alice_pubkey> 0
ScriptPubKey:
    OP_IF // Normal redeem path
        // the owner of <asset2_output> should be Alice
        // which means Alice has redeemed asset2
        <asset2_output> OP_LOOKUP_OUTPUT <Alice_pubkeyhash> OP_EQUALVERIFY 
        OP_DUP OP_HASH160 <Bob_pubkeyhash>
    OP_ELSE // Refund path
        // the premium timelock should be expired
        <locktime> OP_CHECKLOCKTIMEVERIFY OP_DROP
        OP_DUP OP_HASH160 <Alice pubkey hash>
    OP_ENDIF
    OP_EQUALVERIFY
    OP_CHECKSIG
\end{lstlisting}  
\end{figure}

\begin{figure}[htp]
\begin{lstlisting}[
    language=Solidity, 
    basicstyle=\tiny,
    label={lst:bitcoin_contract_option},
    caption={The American Call Option-style Atomic Swap contract in Bitcoin script.}
]
ScriptSig:
    Redeem: <Bob_sig> <Bob_pubkey> 1
    Refund: <Alice_sig> <Alice_pubkey> 0
ScriptPubKey:
    OP_IF // Normal redeem path
        // the owner of the asset2 should not be the contract
        // it should be either (redeemde by) Alice or (refunded by) Bob
        // which means Alice has redeemed asset2
        <asset2_output> OP_LOOKUP_OUTPUT <Alice_pubkeyhash> OP_NUMEQUAL
        <asset2_output> OP_LOOKUP_OUTPUT <Bob_pubkeyhash> OP_NUMEQUAL
        OP_ADD 1 OP_NUMEQUALVERIFY
        OP_DUP OP_HASH160 <Bob_pubkeyhash>
    OP_ELSE // Refund path
        // the premium timelock should be expired
        <locktime> OP_CHECKLOCKTIMEVERIFY OP_DROP
        OP_DUP OP_HASH160 <Alice pubkey hash>
    OP_ENDIF
    OP_EQUALVERIFY
    OP_CHECKSIG
\end{lstlisting}
\end{figure}

% OP_LOOKUP_OUTPUT is from https://bitcoin.stackexchange.com/questions/36229/bitcoin-script-for-a-competitive-crowdfunding-like-contract