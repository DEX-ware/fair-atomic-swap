\section{Formalization}
\label{sec:formalization}

\subsection{American Call Option}

The American Call Option is a contract that ``one can buy an amount of an asset with an agreed price prior to or on an agreed time in the future''.
The agreed price is usually called the \textit{spot price}, and the buying is called \textit{exercising}.
For American Call Option, the buyer can exercise in advanced of the agreed exercise time.
The price of the asset when exercising is called the \textit{strike price}.
Apart from the asset, the option contract itself has value, called the \textit{premium}.
The buyer should pay for both the asset and the premium when participating in the contract.

Note that the price of the asset changes over time. This makes the American Call Option ``speculative'': One may profit from the contract by buying the asset with a lower price then sell it with a higher price.

We define an American Call Option contract $\Pi$ as $(x, y, \pi, S, T, pr)$.
$x$ is the amount of the currency unit (e.g. USD) to participate in the contract.
$y$ is the amount of the asset $\pi$ that the participant hopes to buy.
$S(t)$ is the price of a unit of $\pi$ against the time $t$, and $S(0)$ is the spot price.
$T$ is the agreed strike time.
$pr$ is the \textit{premium}.

\subsection{Atomic Cross-Chain Swap}

(We follow the notion in ``Perun: Virtual Payment Hubs over Cryptocurrencies'')

Alice hopes to buy $x_2$ $Coin_2$ on blockchain $BC_2$ with $x_1$ $Coin_1$ on blockchain $BC_1$.
She initiates an atomic swap, and Bob participates in the atomic swap.
We define an Atomic Swap $\mathcal{AS}$ as $(x_1, Coin_1, x_2, Coin_2)$.

The Atomic Swap protocol consists of the algorithms below:

\begin{enumerate}
    \item \textbf{Setup}: Alice and Bob create addresses on both blockchains.
    \item \textbf{Initiate}: Alice initiates the Atomic Swap by publishing a contract transaction on $BC_1$.
    \item \textbf{Participate}: Bob participates in the Atomic Swap by publishing a contract transaction on $BC_2$.
    \item \textbf{Redeem}: Alice redeems $x_2$ $Coin_2$ and Bob redeems $x_1$ $Coin_1$. Alice should redeem earlier than Bob.
    \item \textbf{Refund}: If Alice or Bob is unsatisfied with the Atomic Swap, he/she can get his/her money back after the timelock of the contract transaction.
\end{enumerate}

\paragraph{Setup}
takes the security parameter $k$,
and returns the address on two blockchains for Alice and Bob $\beta_{A, 1}$, $\beta_{A, 2}$, $\beta_{B, 1}$, $\beta_{B, 2}$.

\paragraph{Initiate}
takes $\beta_{B, 1}$ and $x_1$,
and returns the preimage $s$, the preimage hash $h$, the contract script $\mathcal{C}_1$, the contract transaction $tx_{\mathcal{C}, 1}$, the refund script $\mathcal{R}_1$, and the refund transaction $tx_{\mathcal{R}, 1}$.
The preimage $s$ is a random string generated by Alice. At this stage, $s$ is only known to Alice.
The preimage hash $h = H(s)$, where $H$ is a cryptographic hash function.  $h$ is published when Initiate.
The contract script $\mathcal{C}_1$ is that ``Alice pays $x_1$ $Coin_1$ from $\beta_{A, 1}$ to $\beta_{B, 1}$ if Bob can provide $s$ within a timelock $\Delta_1$. After the time of $\Delta_1$, Alice can refund the money - get $x_1$ $Coin_1$ back.''
The contract script is published as a transaction $tx_{\mathcal{C}, 1}$ on $BC_1$ when Initiate.
The refund script $\mathcal{R}_1$ is that ``Alice pays $x_1$ $Coin_1$ from $\beta_{A, 1}$ to her another address.'' This is to ensure $x_1$ $Coin_1$ can no longer be redeemed by others. Alice can do this only after the timelock $\Delta_1$.
The refund is done by publishing $\mathcal{R}_1$ as a transaction $tx_{\mathcal{R}, 1}$ on $BC_1$ if Alice can and decide to refund.

\paragraph{Participate}
takes $\beta_{A, 2}$, $x_2$ and $h$,
and returns the contract script $\mathcal{C}_2$, the contract transaction $tx_{\mathcal{C}, 2}$, the refund script $\mathcal{R}_2$, and the refund transaction $tx_{\mathcal{R}, 2}$.
The contract script $\mathcal{C}_2$ is that ``Bob pays $x_2$ $Coin_2$ from $\beta_{B, 2}$ to $\beta_{A, 2}$ if Alice can provide $s$ within a timelock $\Delta_2$. Here $\Delta_2$ should expire earlier than $\Delta_1$. After the time of $\Delta_2$, Bob can refund the money - get $x_2$ $Coin_2$ back.''
The contract script is published as a transaction $tx_{\mathcal{C}, 2}$ on $BC_2$ when Initiate.
The refund script $\mathcal{R}_2$ is that ``Bob pays $x_2$ $Coin_2$ from $\beta_{B, 2}$ to his another address.'' This is to ensure $x_2$ $Coin_2$ can no longer be redeemed by others. Bob can do this only after the timelock $\Delta_2$.
The refund is done by publishing $\mathcal{R}_2$ as a transaction $tx_{\mathcal{R}, 2}$ on $BC_2$ if Bob can and decide to refund.

\paragraph{Redeem}
takes $s$,
and returns $\mathcal{V} \in \{true, false\}$ indicating if the redemption is successful or not.
The redemption can be performed by both parties, and Alice should redeem earlier than Bob.
As Alice knows $s$, she can redeem $x_2$ $Coin_2$ - pay $x_2$ $Coin_2$ in $\beta{A, 2}$ to her another address by attaching $s$ in this transaction.
After Alice redeems $x_2$ $Coin_2$, $s$ is published, so that Bob can redeem $x_1$ $Coin_1$, similarly.

\paragraph{Refund}
takes no parameters and returns the $\mathcal{V} \in \{true, false\}$ indicating if the refund is successful or not.
Alice and Bob can perform the refund by publishing $tx_{\mathcal{R}, 1}$ and $tx_{\mathcal{R}, 2}$ after the timelock $\Delta_1$ and $\Delta
_2$, respectively.

% TODO process


\subsection{Atomic Swaps as Premium-free American Call Options}

\begin{enumerate}
    \item \textbf{Advertise}: Alice creates and advertises an American Call Option contract $\Pi = (x_1, x_2, Coin_2, S, \Delta_2, fee)$ by \textbf{initiating} an Atomic Swap $\mathcal{AS}$. Note that $S$ is the exchange rate of $Coin_2 / Coin_1$.
    \item \textbf{Contract}: Bob agrees on $\Pi$ by \textbf{Participating} the $\mathcal{AS}$.
    \item \textbf{Hold}: Alice does not exercise $\Pi$ - \textbf{redeem} the coins from Bob, until $\Delta_2$ is about to expire.
    \item \textbf{Exercise or Abort}: When $\Delta_2$ is about to abort, Alice checks $S$ between $Coin_1$ and $Coin_2$. If she earns money with the latest $S$, she can exercise $\Pi$ by \textbf{redeeming}. Otherwise, she can wait for $\Pi$ to expire and \textbf{refund}.
\end{enumerate}