\section{Formalization}
\label{sec:formalization}

In this section, we formalize the American Call Option and the Atomic Swap protocol,
then prove that the Atomic Swap is equivalent to an American Call Option without the premium.

\subsection{American Call Option}

The American Call Option is a contract that ``one can buy an amount of an asset with an agreed price prior to or on an agreed time in the future''. 
The agreed price is usually called the \textit{spot price}, and the buying is called \textit{exercising}.
For American Call Option, the option buyer can exercise in advanced of the agreed exercise time.
The price of the asset when exercising is called the \textit{strike price}.
As mentioned in Section~\label{subsec:background_option}, the option contract itself has value, and its value is called the \textit{premium}.
The option buyer should pay for both the asset and the premium when participating in the contract.

Note that the price of the asset changes over time. This makes the American Call Option ``speculative'': One may profit from the contract by buying the asset with a lower price then sell it with a higher price.

\begin{definition}
We define an American Call Option contract $\Pi$ as

$$\Pi = (\pi_1, \pi_2, K, A, T, C)$$

where
$\pi_1$ and $\pi_2$ is the currency of the option buyer and the asset of the option seller, respectively; 
$K$ is the strike price with the unit $\pi_2 / \pi_1$ - the price of an unit of $\pi_2$ by using $\pi_1$;
$A$ is the amount of the asset $\pi_2$ that the option seller wants to sell;
$T$ is the agreed strike time;
$C$ is the \textit{premium} with the unit $\pi_1$.
\end{definition}

The process of an American Call Option is as follows:

\begin{enumerate}
    \item \textbf{Advertise}: The option seller creates and advertises an American Call Option contract $\Pi = (\pi_1, \pi_2, K, A, T, C)$.
    \item \textbf{Contract}: The option buyer finds $\Pi$ is profitable, so he participates in $\Pi$.
    To participate, the option buyer should pay $C$ to the option seller first.
    Note that the option buyer does not pay for $A$ $\pi_2$ at this stage.
    Also note that the option seller cannot abort $\Pi$ after the option buyer participates in $\Pi$.
    \item \textbf{Hold}: The option buyer keeps not exercising $\Pi$. If he doesn't exercise $\Pi$ before $T$, $\Pi$ will abort.
    \item \textbf{Exercise}: The option buyer exercises $\Pi$ - The option buyer pays $AK$ $\pi_1$ to the option seller, and the option seller gives $y$ $\pi_2$ to the option buyer. The option buyer can exercise $\Pi$ before or on $T$.
\end{enumerate}

% asset underlying value
The underlying value of $\pi_2$ is fluctuating over time due to the market mechanism.
Intuitively, if the price of $\pi_2$ rises when the option buyer exercises $\Pi$, the option buyer profits, because he buys $\pi_2$ with the price lower than the current market price.
Formally, we denote the asset's underlying value (with the unit $\pi_2 / \pi_1$) at the time $t$ as $S_t$.
Assume the option buyer exercises $\Pi$ at a time $T' \leq T$, he can profit $[(S_{T'} - K) A - C]$ $\pi_1$.
Note that the profit can be negative.
















\subsection{Atomic Swap}

\begin{definition}
We define an Atomic Swap $\mathcal{AS}$ as

$$\mathcal{AS} = (x_1, Coin_1, x_2, Coin_2)$$

where Alice hopes to buy $x_2$ $Coin_2$ on blockchain $BC_2$ from Bob with $x_1$ $Coin_1$ on blockchain $BC_1$.
\end{definition}

The Atomic Swap protocol consists of the algorithms below:

\begin{enumerate}
    \item \textbf{Setup}: Alice and Bob create addresses on both blockchains.
    \item \textbf{Initiate}: Alice initiates the Atomic Swap by publishing a contract transaction on $BC_1$.
    \item \textbf{Participate}: Bob participates in the Atomic Swap by publishing a contract transaction on $BC_2$.
    \item \textbf{Redeem}: Alice redeems $x_2$ $Coin_2$ and Bob redeems $x_1$ $Coin_1$. Alice should redeem earlier than Bob.
    \item \textbf{Refund}: If Alice or Bob is unsatisfied with the Atomic Swap, he/she can get his/her money back after the timelock of the contract transaction.
\end{enumerate}

\paragraph{Setup}
takes the security parameter $k$,
and returns the addresses on two blockchains for Alice and Bob $\beta_{A, 1}$, $\beta_{A, 2}$, $\beta_{B, 1}$, $\beta_{B, 2}$.

\paragraph{Initiate}
takes $\beta_{B, 1}$ and $x_1$,
and returns the preimage $s$, the preimage hash $h$, the contract script $\mathcal{C}_1$, the contract transaction $tx_{\mathcal{C}, 1}$, the refund script $\mathcal{R}_1$, and the refund transaction $tx_{\mathcal{R}, 1}$.
The preimage $s$ is a random string generated by Alice. At this stage, $s$ is only known to Alice.
The preimage hash $h = H(s)$, where $H$ is a cryptographic hash function.  $h$ is published when Initiate.
The contract script $\mathcal{C}_1$ is that ``Alice pays $x_1$ $Coin_1$ from $\beta_{A, 1}$ to $\beta_{B, 1}$ if Bob can provide $s$ before or on a timelock $\Delta_1$ (which is a timestamp). After $\Delta_1$, Alice can refund the money - get $x_1$ $Coin_1$ back.''
The contract script is published as a transaction $tx_{\mathcal{C}, 1}$ on $BC_1$ when Initiate.
The refund script $\mathcal{R}_1$ is that ``Alice pays $x_1$ $Coin_1$ from $\beta_{A, 1}$ to her another address.'' This is to ensure $x_1$ $Coin_1$ can no longer be redeemed by others. Alice can do this only after $\Delta_1$.
The refund is done by publishing $\mathcal{R}_1$ as a transaction $tx_{\mathcal{R}, 1}$ on $BC_1$ if Alice can and decide to refund.

\paragraph{Participate}
takes $\beta_{A, 2}$, $x_2$ and $h$,
and returns the contract script $\mathcal{C}_2$, the contract transaction $tx_{\mathcal{C}, 2}$, the refund script $\mathcal{R}_2$, and the refund transaction $tx_{\mathcal{R}, 2}$.
The contract script $\mathcal{C}_2$ is that ``Bob pays $x_2$ $Coin_2$ from $\beta_{B, 2}$ to $\beta_{A, 2}$ if Alice can provide $s$ before or on a timelock $\Delta_2$ (which is a timestamp). After the time of $\Delta_2$, Bob can refund the money - get $x_2$ $Coin_2$ back.''
Here $\Delta_2$ should expire earlier than $\Delta_1$.
The contract script is published as a transaction $tx_{\mathcal{C}, 2}$ on $BC_2$ when Initiate.
The refund script $\mathcal{R}_2$ is that ``Bob pays $x_2$ $Coin_2$ from $\beta_{B, 2}$ to his another address.'' This is to ensure $x_2$ $Coin_2$ can no longer be redeemed by others. Bob can do this only after $\Delta_2$.
The refund is done by publishing $\mathcal{R}_2$ as a transaction $tx_{\mathcal{R}, 2}$ on $BC_2$ if Bob can and decide to refund.

\paragraph{Redeem}
takes $s$,
and returns $\mathcal{V} \in \{true, false\}$ indicating if the redemption is successful or not.
The redemption can be performed by both parties, and Alice should redeem earlier than Bob.
As Alice knows $s$, she can redeem $x_2$ $Coin_2$ - pay $x_2$ $Coin_2$ in $\beta{A, 2}$ to her another address by attaching $s$ in this transaction.
After Alice redeems $x_2$ $Coin_2$, $s$ is published, \HY{and hence revealed to everyone including Bob, }so that Bob can redeem $x_1$ $Coin_1$, similarly.

\paragraph{Refund}
takes no parameters and returns the $\mathcal{V} \in \{true, false\}$ indicating if the refund is successful or not.
Alice and Bob can perform the refund by publishing $tx_{\mathcal{R}, 1}$ and $tx_{\mathcal{R}, 2}$ after the timelock $\Delta_1$ and $\Delta
_2$, respectively.

Similar to conventional finance assets, $Coin_2$'s value is also fluctuating due to the market mechanism.
We denote the asset's underlying value (with the unit $Coin_2 / Coin_1$) at the time $t$ as $S_t$.














\subsection{Modelling the Atomic Swap as the American Call Option}

We model the Atomic Swap protocol as the American Call Option.
In detail, the Atomic Swap protocol is equivalent to a premium-free American Call Option.

Assume Alice \textbf{initiates} an Atomic Swap $\mathcal{AS} = (x_1, Coin_1, x_2, Coin_2)$ and Bob \textbf{participates} $\mathcal{AS}$.

\begin{theorem}
$\mathcal{AS} = (x_1, Coin_1, x_2, Coin_2)$ is equivalent to the American Call Option contract

$$
\Pi = (Coin_1, Coin_2, \frac{x_2}{x_1}, x_2, \Delta_2, 0)
$$

\end{theorem}


\begin{proof}
% model the atomic swap as aco
In the American Call Option context, the option buyer Alice wants to buy $x_2$ $Coin_2$ from the participant Bob by using $x_1$ $Coin_1$.
$Coin_1$ is the currency Alice uses, $Coin_2$ is the asset Bob has.
$\frac{x_2}{x_1}$ is the price of the asset from Alice's perspective, so $\frac{x_2}{x_1}$.
$\Delta_2$ is the timelock of the contract transaction on $BC_2$.
It is equivalent to the strike time of $\Pi$, because after $\Delta_2$ Bob can refund his asset back and invalidate $\mathcal{AS}$, but before $\Delta_2$ Bob cannot abort $\mathcal{AS}$ if Alice \textbf{participates}.
Establishing the Atomic Swap does not require Alice to pay anything other than the asset to Bob, so the premium here is zero.
\end{proof}













\subsection{Atomic Swap is unfair}

Recall that the premium is the price paid by the option buyer for signing the option contract with the option seller.
In the Atomic Swap, Alice is not required to pay for the premium.
In this way, the initiator can decide whether to proceed the swap while investigating the cryptocurrency market.
This enables the initiator to speculate without any risk: If the initiator's asset price rises right before the strike time, he will proceed the swap to profit, otherwise he will abort the swap to avoid the loss.
Therefore, without the premium, the initiator of an Atomic Swap is risk-free towards the market.

\TODO{maybe some math in this subsec?}