\section{Fair Atomic Swaps}
\label{sec:fair_atomic_swap}

In this section, we propose two fair variants of the original Atomic Swap protocol, of which one is for currency exchange and the other is for American Call Options.

\TODO{Do the formalization as Section3!}


\subsection{Design}

\subsubsection{Difference between Currency Exchange and Options}
\label{subsubsec:diff_spot_option}

Before designing a fair Atomic Swap protocol, we should know its design objective:
Is the protocol aiming at currency exchange or the American Call Options?

To our knowledge, the Atomic Swap protocol is originally designed for the fair exchange between different cryptocurrencies.
However, according to our analysis, the protocol is unfair because Alice can choose to proceed or abort the swap based on the asset price, and he will not receive any penalty if he abort the swap.

The currency exchange and the American Call Option differ in Finance: The currency exchange is a type of Spots~\cite{hull1991introduction}, while the American Call Option is a type of Options.
The Spot Contract and the Option Contract aim at different application scenarios: The Spot Contract aims at exchanging the ownership of assets, while the Option Contract aims at providing the option buyer an ``option'' to trade.
More specifically, Spots and Options differ in the following aspects:

\begin{itemize}
    \item The Spot Contract is exercised immediately, while the Option Contract is exercised on or prior to a specified date in the future.
    \item The Spot Contract cannot be aborted once signed by both parties, while in the Option Contract the option buyer can abort the contract with the loss of the premium.
    \item The Spot Contract itself has no value, while the Option Contract itself has value - the premium.
\end{itemize}

\subsubsection{Atomic Swaps for Currency Exchange and American Call Options}
\label{subsubsec:design_obj}

According to Section~\ref{subsubsec:diff_spot_option}, the currency exchange-style Atomic Swaps and the American Call Option-style Atomic Swaps differ in design objectives.

\paragraph{Atomic Swaps for Currency Exchange}
For the currency exchange, both parties should not abort the contract once signed.
However, in Atomic Swaps, Alice can abort the swap by not releasing the random secret.
Therefore, we should discourage Alice to abort the swap.
To achieve this, we can use the premium mechanism as the collateral: Alice should deposit the premium besides her asset when \textbf{Initiate}($\cdot$).
The premium should follow that:
\textbf{Alice pays the premium to Bob if Bob refunds his asset after his timelock but before Alice's timelock.
If Alice's timelock expires, Alice can refund her premium back.}

\paragraph{Atomic Swaps for American Call Options}
For the American Call Options, the option buyer should pay for the premium besides the bought asset, regardless whether the contract is settled or aborted.
In reality, the option sellers are trustworthy - They never abort the contract.
However, in Atomic Swaps, Bob can abort the contracts like Alice.
To keep the Atomic Swap consistent with the American Call Options,
the premium in American Call Option-style Atomic Swaps should follow that: 
\textbf{Alice pays the premium to Bob if
1) Alice redeems Bob's asset before Bob's timelock, or
2) Bob refunds his asset after Bob's timelock but before Alice's timelock.
If Alice's timelock expires, Alice can refund her premium back.}











\subsection{Our protocols}
We propose two fair Atomic Swap protocols based on the original Atomic Swap protocol, but apply our design objectives in Section~\ref{subsubsec:design_obj}.
One of the two protocols is for currency exchange, and the other is for American Call Options.
The two protocols are similar to each other, and only differ on the rule of premium. 

Formally, define an Atomic Swap protocol $\mathcal{AS}'$ as

$$\mathcal{AS}' = (x_1, Coin_1, x_2, Coin_2, pr)$$

where $pr$ is the amount of the premium measured in $Coin_2$.
In our protocols, besides $x_1$ $Coin_1$, Alice should also lock $pr$ $Coin_2$ on $BC_2$, which will be described later.



Similar to the original Atomic Swap $\mathcal{AS}$, our protocol contains the five algorithms
\textbf{Setup}($\cdot$), \textbf{Initiate}($\cdot$), \textbf{Participate}($\cdot$), \textbf{Redeem}($\cdot$) and \textbf{Refund}($\cdot$).
\textbf{Setup}($\cdot$) in $\mathcal{AS}'$ is the same as in $\mathcal{AS}$, but the rest algorithms are different. 

\paragraph{\textbf{Initiate}($x_1, sk_{A, 1}, \beta_{B, 1}, sk_{A, 2}, x_2, \beta_{B, 2}, s, \delta_1, \delta_2$)}
\textbf{Initiate}($\cdot$) can only be invoked by Alice, which is guaranteed by the ownership of $sk_{A, 1}$ and $sk_{A, 2}$.
In addition to $\mathcal{AS}$, \textbf{Initiate}($\cdot$) further takes the secret key $sk_{A, 2}$ of Alice on $BC_2$,
the amount $x_2$ of $Coin_2$ Bob wishes to exchange,
Bob's address $\beta_{B, 2}$ on $BC_2$,
and the timelock $\delta_2$ for Bob (chosen by Alice).
In addition to $\mathcal{AS}$, \textbf{Initiate}($\cdot$) further returns Bob's contract script $\mathcal{C}_2$, and
the associated transaction $tx_{\mathcal{C}, 2}$.

$\mathcal{C}_1$ and $tx_{\mathcal{C}, 1}$ is the same as in $\mathcal{AS}$, while $\mathcal{C}_2$ (as well as $tx_{\mathcal{C}, 2}$) is more sophisticated here - It contains two coherent sub-contracts as follows:

\begin{itemize}
    \item $\mathcal{C}^{asset}_2$: The contract for the asset $x_2$ $Coin_2$
    \item $\mathcal{C}^{pr}_2$: The contract for the premium $pr$
\end{itemize}

$\mathcal{C}^{asset}_2$ is the same as in $\mathcal{AS}$.
$\mathcal{C}^{pr}_2$ differs for the currency exchange and the American Call Options.

The content of $\mathcal{C}^{pr}_2$ for currency exchange-style Atomic Swap $\mathcal{AS}'_{c}$ and American Call Option-style Atomic Swap $\mathcal{AS}'_{o}$ are shown below:

\begin{description}
    \item[$\mathcal{C}^{pr}_2$ in $\mathcal{AS}'_{c}$] Alice pays $pr$ to Bob with the condition:
    Bob refunds $x_2$ $Coin_2$ after $\delta_2$ and before $\delta_1$.
    If $\delta_1$ expires, Alice can refund $pr$ back.
    \item[$\mathcal{C}^{pr}_2$ in $\mathcal{AS}'_{o}$] Alice pays $pr$ to Bob with one of the two conditions:
    1) Alice redeems $x_2$ $Coin_2$ before $\delta_2$.
    2) Bob refunds $x_2$ $Coin_2$ after $\delta_2$ but before $Delta_1$ (note that $\delta_2 < \delta_1$).
    If $\delta_1$ expires, Alice can refund $pr$ back.
\end{description}

Alice published $tx_{\mathcal{C}, 1}$ on $BC_1$ and $tx_{\mathcal{C}, 2}$.
Note that Alice only triggers $\mathcal{C}_1$ and $\mathcal{C}^{pr}_2$ to execute at this stage.
Bob can trigger $\mathcal{C}^{asset}_2$ to execute by \textbf{Participate}($\cdot$) later.


\paragraph{\textbf{Participate}($sk_{B, 2}, tx_{\mathcal{C}, 1}, tx_{\mathcal{C}, 2}$)}
\textbf{Participate}($\cdot$) can only be invoked by Bob, which is guaranteed by the ownership of $sk_{B, 2}$.
It takes Bob's secret key $sk_{B, 2}$ on $BC_2$,
Alice's contract transaction $tx_{\mathcal{C}, 1}$,
and Bob's contract transaction $tx_{\mathcal{C}, 2}$.
It returns $\mathcal{V} \in \{0, 1\}$ indicating if the participation is successful or not,
the refund contract $\mathcal{R}_2$,
and the refund transaction $tx_{\mathcal{R}, 2}$.

At this stage, Bob deposits $x_2$ $Coin_2$ in $\mathcal{C}^{asset}_2$, and triggers $\mathcal{C}^{asset}_2$ to execute.
Note that Bob can decide whether to participate in $\mathcal{AS}'$ by auditing $tx_{\mathcal{C}, 1}$ and $tx_{\mathcal{C}, 2}$: If both contracts are fair, Bob will trigger \textbf{Participate}($\cdot$), otherwise Bob will not and find more profitable contracts from others.
$tx_{\mathcal{R}, 2}$ in $\mathcal{AS}'$ is the same as in $\mathcal{AS}$.

\paragraph{\textbf{Redeem}($sk_{id, i}, s, tx_{\mathcal{C}, 1}, tx_{\mathcal{C}, 2}$)}
\textbf{Redeem}($\cdot$) can be invoked by both parties, but with the condition $(id = A \wedge i = 2) \vee (id = B \wedge i = 1)$.
% input
It takes the secret key $sk_{id, i}$,
Alice's random secret $s$,
and both contract transaction $tx_{\mathcal{C}, 1}$ and $tx_{\mathcal{C}, 2}$.
% output
It returns $\mathcal{V} \in \{0, 1\}$ indicating if the redemption is successful or not.


Redeeming $x_1$ $Coin_1$ for Alice and $x_2$ $Coin_2$ for Bob in $\mathcal{AS}'$ are the same as in $\mathcal{AS}$.
Different from $\mathcal{AS}$, $\mathcal{C}^{pr}_2$ in $tx_{\mathcal{C}, 2}$ will work once triggering \textbf{Redeem}($\cdot$) in $\mathcal{AS}'$.

\paragraph{\textbf{Refund}($sk_{id, i}, tx_{\mathcal{R}, i}$)}
\textbf{Refund}($\cdot$) can be invoked by both parties, but with the condition $(id = A \wedge i = 1) \vee (id = B \wedge i = 2)$.
It takes the secret key $sk_{id, i}$,
and the refund transaction $tx_{\mathcal{R}, i}$.
It returns the $\mathcal{V} \in \{0, 1\}$ indicating if the refund is successful or not.

Refunding $x_2$ $Coin_2$ for Alice and $x_1$ $Coin_1$ for Bob are the same as in $\mathcal{AS}$.
Different from $\mathcal{AS}$, $\mathcal{C}^{pr}_2$ in $tx_{\mathcal{C}, 2}$ will work once triggering \textbf{Refund}($\cdot$).