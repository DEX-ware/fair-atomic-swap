\section{A Fair Atomic Cross-Chain Swap}
\label{sec:fair_atomic_swap}

We propose a solution to make the Atomic Cross-Chain Swap fair: 
The initiator is responsible for creating the contracts on both blockchains,
and the participant can choose whether to participate in the swap or not.
In addition to the original protocol, the initiator should lock an amount of the premium on $BC_2$ as the cost of aborting the swap.
If the swap is successful, the initiator can redeem the premium back, otherwise the participant will redeem the premium.

\subsection{Design}

\subsubsection{Difference between Currency Exchange and Options}

% option: premium goes to seller (participant)
%   if swap is successful || initiator aborts the swap
% currency exchange: premium goes to seller 
%   if initiator aborts the swap

\subsubsection{Atomic Swaps for Currency Exchange and Options}

% design objecive: fairness
% 1. cannot get both assets
% 2. fair exchange of assets
%   (how to define fair? currency exchange: no premium
%    option: must pay premium
% 3. initiator cannot painlessly abort
% 4. participant can choose to or not to join, but cannot abort

% why premium should be on blockchain2

% timing constraint for each step

\subsection{Our Protocol Construction}

We define an Atomic Cross-Chain Swap protocol $\mathcal{AS}'$ as

$$\mathcal{AS}' = (x_1, Coin_1, x_2, Coin_2, pr)$$

where $pr$ is the premium amount that the initiator should lock on $BC_2$, measured in $Coin_2$.

The protocol follows the steps below:

\begin{enumerate}
    \item \textbf{Setup}: Alice and Bob create addresses on both blockchains.
    In addition, Alice creates both contracts $\mathcal{C}_1$ and $\mathcal{C}_2$ and publishes them as transactions $tx_{\mathcal{C}, 1}$ and $tx_{\mathcal{C}, 2}$ on $BC_1$ and $BC_2$. 
    \item \textbf{Initiate}: Alice locks her $x_1$ $Coin_1$ on $BC_1$, and lock her $pr$ on $BC_2$.
    \item \textbf{Participate}: Bob locks his $x_2$ $Coin_2$ on $BC_2$. Note that Bob's timelock expires earlier than Alice's.
    \item \textbf{Redeem}: Alice redeems $x_2$ $Coin_2$ and Bob redeems $x_1$ $Coin_1$. Alice should redeem earlier than Bob.
    \item \textbf{Refund}: If Alice does not redeem $x_2$ $Coin_2$ before Bob's timelock expires, Bob can refund $x_2$ $Coin_2$ and redeem $pr$.
    If Bob does not redeem $x_1$ $Coin_1$ before Alice's timelock expires, Alice can refund $x_1$ $Coin_1$ and $pr$.
\end{enumerate}

\paragraph{Setup}
In addition to the address generation in the original protocol,
Alice creates and publishes both contracts on $BC_1$ and $BC_2$.
In the original protocol, Alice and Bob create $\mathcal{C}_1$ and $\mathcal{C}_2$, and publish $tx_{\mathcal{C}, 1}$ and $tx_{\mathcal{C}, 2}$ on $BC_1$ and $BC_2$, respectively.
In our protocol, Alice is responsible for creating and publishing both of them.

$\mathcal{C}_1$ remains the same as the original protocol, while $\mathcal{C}_2$ is more sophisticated here - It contains two coherent sub-contracts as follows:

\begin{itemize}
    \item $\mathcal{C}^{asset}_2$: The contract for the asset $x_2$ $Coin_2$
    \item $\mathcal{C}^{pr}_2$: The contract for the premium $pr$
\end{itemize}

$\mathcal{C}^{asset}_2$ is the same as in the original protocol。
$\mathcal{C}^{pr}_2$ differs for the currency exchange and the American Call Options.

For currency exchange, $\mathcal{C}^{pr}_2$ is that
``Alice pays $pr$ to Bob with the condition: If Bob refunds $x_2$ $Coin_2$ before the timelock $\Delta_1$ (which is definite as $x_2$ $Coin_2$ is timelocked by $\Delta_2 < \Delta_1$), Bob can redeem $pr$, otherwise Bob cannot.
If $\Delta$ expires, Alice can refund $pr$ back.''

For American Call Options, $\mathcal{C}^{pr}_2$ is that
``Alice pays $pr$ to Bob with the condition: If Alice redeems $x_2$ $Coin_2$ before the timelock $\Delta_1$ (which is definite as $x_2$ $Coin_2$ is timelocked by $\Delta_2 < \Delta_1$), Bob can redeem $pr$, otherwise Bob cannot.
If $\Delta$ expires, Alice can refund $pr$ back.''

% ``
% if $\Delta_1$ is expired:
%   initiator can refund $pr$
% else:
%   initiator cannot refund $pr$
%   if asset2 is refunded: // for currency exchange
%   if asset2 is redeemed: // for options
%     participant can redeem $pr$
%   else:
%     participant cannot redeem $pr$
% ''

Note that $\Delta_1$ in $\mathcal{C}_1$ is the same as in $\mathcal{C}^{pr}_2$, and $\mathcal{C}^{asset}_2$ is with the timelock $\Delta_2$.

Also note that neither of them triggers contracts at this stage.
Instead, Alice and Bob can enter the contracts in the future.
In this stage, Bob can decide whether to participate in $\mathcal{AS}'$ by auditing $tx_{\mathcal{C}, 1}$ and $tx_{\mathcal{C}, 2}$: If both contracts are fair, Bob will participate, otherwise Bob will not participate in this contract and find contracts from others.

\paragraph{Initiate}
Alice initiates the swap by entering $\mathcal{C}_1$ and $\mathcal{C}^{pr}_2$.
First, Alice deposits her $x_1$ $Coin_1$ in $\mathcal{C}_1$, and triggers the hash-locked payment to Bob.
Second, Alice deposits her $pr$ in $\mathcal{C}^{pr}_2$, and triggers the conditional payment to Bob.
Note that Alice can specify an arbitrary amount for $pr$ when creating $\mathcal{C}_2$, and Alice should have enough $Coin_2$ to deposit $pr$.

\paragraph{Participate}
If Bob thinks $\mathcal{AS}'$ is fair for him, he will participate in $\mathcal{AS}'$ by depositing $x_2$ $Coin_2$ in $\mathcal{C}^{asset}_2$, and triggers the the hash-locked payment to Alice.


\paragraph{Redeem}
Same as the original protocol, Alice redeems $x_2$ $Coin_2$ in $\mathcal{C}^{asset}_2$ by releasing $s$, then Bob can redeem  $x_1$ $Coin_1$ in $\mathcal{C}_1$ using the released $s$.
Note that once Bob redeems $x_1$ $Coin_1$, he cannot redeem $pr$ anymore according to $\mathcal{C}^{pr}_2$.

\paragraph{Refund}
Refunding $x_2$ $Coin_2$ for Alice and $x_1$ $Coin_1$ for Bob are the same as in the original protocol.
Alice can refund $pr$ after $\Delta_1$ if Bob has not refunded $x_1$ $Coin_1$.
If Bob refunds $x_1$ $Coin_1$ before $\Delta_2$ (which is definitely before $\Delta_1$ as $\Delta_2 < \Delta_1$), Bob can also redeem $pr$.

