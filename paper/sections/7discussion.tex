\section{Discussion}
\label{sec:discussion}

\subsection{Security of Atomic Cross-Chain Swaps}

Atomic Cross-Chain Swaps are not perfect, and have some security issues.

% rely on blockchain security
First, the security of Atomic Cross-Chain Swaps relies on the security of blockchains:
If the blockchains involved in the swaps are insecure, the Atomic Cross-Chain Swaps are also insecure.

% smart contract /script
Second, the Atomic Cross-Chain Swap contracts are written by high-level languages.
This means the compiled contracts can be insecure if the contract compilers are flawed.

% 2 blockchain async
Furthermore, the timing mechanism may not be reliable, while the Atomic Swaps are based on timelocks.
% timestamp of blockchain
In blockchains, events are timestamped by either two approaches: The relative time or the absolute time.
% relative time
The relative time uses the block height to represent the time.
A block height cannot represent a precise time, as the new block generation is a random process.
% abosolute time
The absolute time uses the UNIX timestamp, and the consensus participants are responsible for timestamping blocks.
However, the consensus participants may use the wrong time, either by purpose or by accident.
% async blockhains
Furthermore, blockchains are independent to each other, which means they are asynchronous in terms of time.
This can lead to the inconsistency on the timelocks of $Coin_1$ and $Coin_2$.


\subsection{Other Countermeasures for the Unfair Atomic Cross-Chain Swaps}

1.  We could force that setting up the HTLCs requires payment.

2.  Exchange nodes could increase their fees.

3.  Exchange nodes could limit the timelock of cross-asset swaps.


\subsection{Limitations of Our Solutions}

% need some asset2 to do atomic swap

% not instant

%...